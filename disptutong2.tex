\section{Tutkimusongelma}
\label{disptutong1}

\begin{enumerate} %%1
\item Tutkimusongelma\\

Tutkielman ydin on esitellä kahden luokittelumuuttujan korrespondenssianalyysi ja sen eräitä laajennuksia. Painopiste on graafisten menetelmien esittely esimerkkiaineistolla, joka on riittävän laaja ja monipuolinen. Yksinkertainen kahden luokittelumuuttujan korrespondenssianalyysi antaa graafisen analyysin ``...perussäännöt tulkinnalle. Kaikki muut korrespondenssianalyysin muodot ovat saman algoritmin soveltamista toisen tyyppiisiin datamatriiseihin, ja tulkintaa sovelletaan vastaavasti (with the consequent adaptation of the interpretation''( \cite{RefWorks:57}, s. 437).

Tulkinnat eivät kuitenkaan ole aivan yksinkertaisia. Greenacre ehdottaa artikkelissaan (\cite{RefWorks:73}, s. 20- ) viittä sääntöä yksinkertaisen korrespondenssianalyysin ongelmakohtien välttämiseen. Ne liittyvät oleellisesti graafisen analyysin tulkintapulmiin ja ns. skaalausongelmaan (symmetriset ja epäsymmetrisen kuvat, standardi- ja pääkooridinaatit, standard and principal coordinates) .Näistä perusasioista käydään edelleen keskustelua (kts. esim. \cite{RefWorks:77}).

Graafiset menetelmät ovat myös taitolaji, kuvissa on askel kerralaan päästävä kokeiluista kohti selkeää esitystä jossa toivon mukaan on vain oleelllinen informaatio (\cite{RefWorks:60}.\\

Korrespondenssianalyysi sopii kaikkien sellaisten aineistojen analyysiin, joissa data voidaan esittää jonkinlaisina tutkimusongelman kannalta järkevinä lukumäärinä. Teknisesti taulukon lukujen on oltava ei-negatiivisia (tosin LeRoux et. al. sallivat myös negatiiviset luvut tietyin ehdoin \cite{RefWorks:68} , s. 60) ja mitta-asteikon on oltava sama (lukumääriä, metrejä, euroja jne). Luokitteluasteikko on tavallaan hyvin lievä oletus, ja menetelmä sovelutuu mainiosti esimerkiksi järjestysasteikon muuttujan analyysiin. Sen avulla nähdään selvemmin kyselytutkimusten Likert-asteikon ``oikea'' mitta-asteikko, useinhan se oletetaan tasaväliseksi. Yksinkertaisen korrespondenssianalyysin laajennukset mahdollistavat mys monipuolisempia tutkimusasetelmia ja lisäinformaation käytön.

\bigskip

\item Tavoitteet
\begin{enumerate} %%2
\item Historiaa tiiviisti
Sivuutan laajemman menetelmän historian käsittelyn, mutta sanon siitä toki jotain. Historiaa esitellään lyhyesti ja yleisesti \cite{RefWorks:68}. Kahman väitöskirjan \cite{RefWorks:74} johdantoluku kertoo menetelmän käytöstä suomalaisessa sosiologiassa. Kahden luokittelumuuttujan korrespondenssianalyysistä löytyy (simple ca) bibliografinen katsaus vuodelta 2004 \cite{RefWorks:70}. CA ja sen kaltaiset menetelmät -- lähisukulaiset -- ovat  olleet käyt;ssä monilla tieteenaloilla. On tavallaan keksitty muutaman kerran uudestaan (Japanissa, ``homogeenisyysanalyysinä (Gifi)'' , ``spektirianalyysinä'') ja sillä on yhteyksiä myös ns. ``compositiona data (koostumus-data?)'' -menetelmiin (\cite {RefWorks:55}). Taustalla on melko yleinen data-analyysin tutkimusongelma.

\item Yksinkertaisen korrespondenssianalyysin esittely

Tavoite on esitellä vaiheittain korrespondenssianalyysin peruskäsitteet ja graafisen esitykset niin, että ne voi ymmärtää perehtymättä vektoriavaruuksien isomorfimismeihin tai matriisialgebran perusteisiin. Samalla menetelmän matemaattinen rakenne kuvataan riittävän täsmällisesti, jotta siihen voidaan viitata kun graafisen menetelmien pulmakohtia selvennetään. Matemaattisen osan tavoite on myös auttaa näkemään korrespondenssianalyysi eräänä monimuuttujamentelmänä, ja yhtenä tapana analysoida moniulotteisia aineistoja. Graafinen data-analyysi vaatii my;s kuvien muokkaamista, ja silloin on kyettävä hahmottamaan korrespondenssianalyysin tulostietojen sisält;.

\end{enumerate} %%2
\bigskip
\item Rajaus - mitä ei tutkita)
\begin{enumerate} %%3
\item ekologisessa tutkimuksessa suosittu ``detrended ca''
\item oppihistoria, erityisesti CA:n toisistaan riippumattomat ``keksimiset''
\item Bourdieun tutkimuset ja niiden inspiroima kansainvälinen tutkimus, joka jatkuu edelleen
\item laskennalliset (algoritmi) ongelmat, esim. harvojen (sparse) data-matriisien käsittely
\item sovellusalueet ohitan vain maininnalla (biologia, arkeologia, lääketiede, kirjallisuus/kielitiede)
\item Otantatutkimus ja korrespondenssianalyysi\\

Perinteisesti korrespondenssianalyysissä on korostettu jyrkkää eroa ``todennäköisyysteoreettiseen'' tilastolliseen päättelyyn, ja tätä vastakkainasettelua pitää hieman kuvata. Monissa tapauksissa todennäköisyysteoreettiset käsitteet tulevat mukaan analyysiin, mutta sivuutan ainakin otantatutkimuksen (esim. analyysipainojen käyttö).
\end{enumerate} %%3

\item Puuttuva osa - data-analyysin lähestymistavat

Jätän viimeiseksi aiheen, joka on kiinnostava mutta vaikeasti hahmotettava. Korrespondenssianalyysi kuuluu tilastollisten menetelmien joukossa eksploratiivisiin (vs. konfirmatorinen), ei-parametrisiin mentelmiin. On katsottava dataa, datan ehdoilla. Tämä korostus on ollut vahva menetelmän ranskalaisilla kehittäjillä, ja ehkä muillakin. Jotain tästä pitää sanoa, mutta mitä?

\end{enumerate} %%1



