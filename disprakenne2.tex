\section{Tutkielman rakenne ja esitystapa}
\label{disprakenne1}

\textbf {Tutkielman sisältö ja rakenne}

\begin{enumerate} %%1

\item Yksinkertainen korrespondennsianalyysi

\begin{enumerate} %%2

\item Johdatteleva esimerkki - kahden luokittelumuuttujan frekvenssitaulukko

- peruskäsitteet: lähteinä MG:n oppikirjat ja luentomateriaalit, sovelutuvin osin myös \cite{RefWorks:68}. \\

- CA:n oletuskartta (symmetrinen kartta)\\

- tulkinnan perussäännöt; yksinkertainen ``rautalanka-metodologia'' tiiviisti (\cite{RefWorks:68}).\\
- ``CA jargon'': profiilit, massat, inertia, modaliteetit\\

\item Varianssianalyysi - yksiulotteinen korrespondenssianalyysi
- ehkä hyvä didaktinen keino

\item Kaksoiskuvat (biplots) ja korrespondenssianalyysi

Korrespondenssianalyysin pääväline -- kartta -- ei ole aivan yksinkertainen, ja tulkinnoissa voi ajautua harhateille. Korrelaatiodiagrammaa muistuttavaan kaksiulotteiseen kuvaan on projisoitu kaksi pistepilveä, ja kartan ``lukeminen'' on pisteryhmien välisten ja sisäisten etäisyyksien oikeaa ymmärrystä.\\

Kaksoiskuva (biplot) on yleinen geometrinen menetelmä havaintojen ja muuttujien graafiseen esittämiseen samassa kuvassa. Korrespondenssianalyysin kartat ja niiden rajoitukset on hyvä ymmärtää tässä yleisemmässä kehikossa. Lähteinä (\cite{RefWorks:56}, \cite{RefWorks:73}.\\
Tämä jakso voisi olla omana kokonaisuutena?\\
Tämän osuuden jälkeen seuraavassa jaksossa sovelletaan näitä käsitteitä data-analyysissä.

\end{enumerate}

\item Yksinkertaisen korrespondenssianalyysin laajennuksia

 Tässä jaksossa täydennetään alustavaa esimerkkianalyysiä, ja samalla esitellään korrespondenssianalyysin graafisia menetelmiä laajemmin. Konribuutio-kuvat ja inertian dekomponointi ennen muuta, mutta kuvien ``säätämisessä'' on myös paljon vaihtoehtoja.\\
 
 Tärkeimmät lähteet \cite{RefWorks:79} ja \cite{RefWorks:55}.
 
\begin{enumerate} %%3

\item Täydentävät muuttujat (supplementary points)
\item ''Stacked and concatenated tables/matrices''
\item ``ABBA''
\item Osajoukon korrespondenssianalyysi (subset CA)

- 

\end{enumerate} %%3

\item Usean muuttujan korrespondenssianalyysi

\begin{enumerate} %%4
 
\item Indikaattorimatriisi ja Burtin matriisi
 - pulmia tulkinnassa ja tuloksissa
\item JCA - Joint Correspondence Analysis

\end{enumerate} %%4
 
\item Korrespondenssianalyysi regressiomallin kaltaisessa tutkimusasetelmassa (kanoninen CA) 
 Suosittu ympärist- ja biotieteissä, esimerkiksi lajiston esiintymisdataa jota halutaa analysoida ``regressiotyyliin'' ehdollisesti selittavien (mahdollisesti jatkuvien) muuttujien  suhteen (\cite{RefWorks:78} ja \cite{RefWorks:55}).
 
\item Korrespondenssianalyysi ja koostumusdata (compositional data)

 Nämä kaksi jälkimmäistä eivät nyt ehkä ole ihan loppuun asti harkittuja. MG:n kirjasta löytyy muutakin.
 
\item Matched matrices

\item Square tables

\item Korrespondenssianalyysin ja muiden monimuuttujamenetelmien yhteyksistä

\begin{enumerate} %%5

\item Korrespondenssianalyysi ja pääkomponenttianalyysi
\item Korrespondenssianalyysi ja moniulotteinen skaalaus
\item Luokittelu- ja erottelumenetelmät

- lähteinä Mustosen ja Vehkalahden kirjat ( \cite{RefWorks:62}, \cite{RefWorks:76}).
- vain lyhyesti, koska joskus CA:n graafisista esityksistä virheellisesti ``tunnistetaan'' ryhmiä tms.

\end{enumerate} %%5

\end{enumerate} %%1




