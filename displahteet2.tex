\section{Lähdeaineisto ja data}
\label{displahteet1}

Koitan pitää lähdeluettelon kohtuullisen kokoisena.

\subsection{Kirjallisuus ja artikkelit}

Tutkielman tärkein lähdeaineisto ovat Greenacren oppikirja \cite{RefWorks:56} ja muut artikkelit. Julkaisu kaksoiskuvista (biplots) \cite{RefWorks:55} esittää korrespondenssianalyysille oleelliset kaaviot osana yleisempää graafista analyysitapaa. \\

Käytän keväällä 2017 HY:ssa järjestetyn kurssin luentokalvoja, ainakin muistilistan tapaan. Niissä ei liene mitään uutta, joten viittaukset ovat kirjallisiin
lähteisiin. LeRouxin ja Rouanetin \cite{RefWorks:68} kirja esittelee menetelmää ranskalaisen perinteen mukaisesti, uudella otsikolla (geometrinen data-analyysi). Uppsalan luentokalvot (LeRoux) ovat hyvä tiivistys. En tiedä kuinka laajasti tätä selostan, mutta metodologiset ideat ja metodin käytännön ohjeet kannattaa ainakin poimia.
Koitan ainakin mainita muutaman suomalaisen tutkimuksen, ja viitata muutamaan löytämääni luetteloon. Kulttuurisia eroja Suomessa on tutkittu ranskalaisten esikuvien (Bourdieu) tyyliin kartan käsittein (cultural map) \cite{RefWorks:72}, ja Nina Kahman väitöskirjassa ''Yhteiskuntaluokka ja maku'' (\cite{RefWorks:74} sovelletaan myös useamman muuttujan korrespondenssianalyysiä (MCA). Jari Oksanen (http://cc.oulu.fi/~jarioksa/) on julkaissut biotieteiden R-paketteja, ja osallistunut myös keskusteluun ns. ``detrented CA'' - menetelmän hyvistä ja huonoista puolista (kts. esim.  \cite{RefWorks:70} ja siinä mainitut lähteet). Olen silmäillyt myös joitakin kotimaisia pro gradu-tutkielmia, mutta jätän ne tässä vaiheessa sivuun.




\subsection{Data ja R-koodi}

Aineisto on ISSP:n viimeisin (2012) ``Family and Changing Gender Roles'' - data. Se antaa hyvän mahdollisuuden laajentaa analyysiä yksinkertaisesta yhä isompaan ja mutkikkaampaan. Sama data vuodelta 2002 oli käytössä kevään 2017 MCA-kurssilla. Aion käyttää myös kurssin laskuharjoitusten r-koodia mallina.\\

Aineisto on vapaasti ladattavissa, mutta lähdeviitteistä se vielä puuttuu. 

Ohjelmisto on R, en esittele muita laajasti käytettyjä tilasto-ohjelmia (esim. SAS,
SPSS). Pieni viite kuitenkin, kun Survo tässäkin oli aikanaan eturivissä.

\bigskip

Kirjoitan tutkielman LateX:lla, ja data-analyysin Rmarkdown-raportteina. Tässä kokeilen ja haen vielä oikeita asetuksia ympäristölle.

\bigskip

Omia muistiinpanoja: \\

ISO-8859-1 vai UTF8, doctype (raport vai article) jne.\\

Kuinka tarkasti on kuvattava datan muunnokset sopivaan R-muotoon?\\

Selvitettävä Rmarkdown -> LateX - yhdistelmä - ja tätä päätettiin käyttää (1.9.2018). Korjailin vielä vihoviimeisen kerran skandeja.\\